% getting_started.tex

\chapter{Getting Started}
This chapter provides a short real-world example of performing symbolic regression with \RGP.

\section{Installation}

The newest release version of \RGP is always available on CRAN. To install it just type
\lstinline!install.packages(``rgp'')! 

\section{Problem Definition}

For a first run of \RGP you need to define a fitting problem.
\begin{lstlisting}[caption = {Creating a function with variables}, label = lstExample]
makeDampedPendulum <- function(A0 = 1, g = 9.81, l = 0.1, phi = pi, gamma = 0.5) {
  omega <- sqrt(g/l)
  function(t) A0 * exp(-gamma * t) * cos(omega * t + phi)
}
\end{lstlisting}

The function above represents a damped mathematical Pendulum, the arguments are the starting Amplitude A0, 
gravity g, length of pendulum l, phase, radial frequency omega and damping gamma.

\begin{lstlisting}[caption = {Attributes}, label = tutattributes]
dampedPendulum1 <- makeDampedPendulum(l = 0.5)
\end{lstlisting}

Here is a example of alternating the given attributes, 
the lenght is altered and the function is assigned to a new name.

\begin{lstlisting}[caption = {Sequence Creating }, label = tutsequence]
xs1 <- seq(from=1, to=10, length.out=512)
\end{lstlisting}

A list of numbers is definded for creating a dataframe. It creates a subsequent list of numbers from 1 to 10
with 512 steps.

\begin{lstlisting}[caption = {Data Frame Creation}, label = tutdataframe]
dampedPendulumData <- data.frame(time=xs1,
  amplitude=dampedPendulum1(xs1) + rnorm(length(xs1), sd=0.01))
\end{lstlisting}

We create a time series dataframe with the sequence defined above. 
The dataframe consists of the sequence xs1 as time, the amplitude defined through our pendulum-function and a
interference function.
To plot the prepared data frame type \lstinline!plot(dampedPendulumData, pch=20)!.

Now let's shift attention to the more interesting objects.
First, we need to activate \RGP using \lstinline!require(rgp)!.
We are going to perform a symbolic regression with our pendulum data.
The time permitted for the regression are 2 minutes ( 2 * 60 seconds ).
To start the symbolic regression you need the following code.

\begin{lstlisting}[caption = {Symbolic Regression }, label = tutsymbolicregression]
modelSet1 <- symbolicRegression(amplitude ~ time, data=dampedPendulumData,
                                stopCondition=makeTimeStopCondition(2 * 60))
\end{lstlisting}

Now we got a set of models with variable fitness. 
To get our desired prediction we are going to use the model with the best training fitness
which gives the best results.

\begin{lstlisting}[caption = {Best Model}, label = tutbestmodel]
bestModel1 <- modelSet1$population[[which.min(Map(modelSet1$fitnessFunction, 
  modelSet1$population))]]

bestModel1
\end{lstlisting}

Let's start the prediction and plot our given and predicted data. 

\begin{lstlisting}[caption = {Prediction}, label = tutPrediction]

predictedData <- data.frame(time=xs1,
  amplitude=predict(modelSet1, newdata=dampedPendulumData))

#plot data
plot(dampedPendulumData, pch=20)
points(predictedData, pch=20, col=2)

\end{lstlisting}



\subsection{An Example for the Listings Package} % TODO nur ein Beispiel, bitte spaeter entfernen!
Listing \ref{lstExample} shows an example on how to integrate \R code listings into \LaTeX~documents
using the nice {\sf Listings} package\footnote{See
  \url{ftp://ftp.tex.ac.uk/tex-archive/macros/latex/contrib/listings/listings.pdf} for further
  information.}. As you might have noticed, the last sentence also demonstrates how to refer to
listings via the {\tt label}/{\tt ref} pair of \LaTeX~commands.

\begin{lstlisting}[caption = {test}, label = lstExample]
makePendulum <- function(A = 1, g = 9.81,
                         l = 0.1, phi = pi) {
  omega <- sqr(g / l)
  function(time) A * cos(omega * time * phi)
}
\end{lstlisting}

The {\sf Listings} package also supports typesetting of small code snippets like
\lstinline!gpModel$fitness! through its {\tt lstinline} command. Please note the slightly peculiar way
the snippets are delimited in the \LaTeX~source code. You can use any character that does not occur
in the snippet as a delimiter. We are using the {\tt !} character in this example.
